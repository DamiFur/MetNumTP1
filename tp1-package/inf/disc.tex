\section{Discusi\'on}

\subsection{Heur\'istica para obtener mayor posici\'on posible}

El enunciado nos pide pensar una estrategia (o heur\'istica) para obtener la mayor posici\'on posible, buscando minimizar el n\'umero de partidos ganados. La entrada de nuestro algoritmo, son todos los partidos en la competencia y un equipo en particular en el cual podemos decidir si un partido lo gan\'o o lo perdi\'o. Luego de ver los distintos resultados que tuvimos analizando casos, se nos ocurri\'o las siguientes dos heur\'isticas que compararemos entre si:\\
\\
(A) Para cada partido, si no soy el mejor equipo de todos en el ranking final, gano el partido. Si soy el mejor del ranking, lo pierdo.\\
(B) Para cada partido, si el equipo con el que juego esta en un ranking superior al mio le gano, sino me dejo perder.\\
\\
Para medir el ranking utilizaremos el metodo de Colley para ver como se desempe\~nan nuestros algoritmos. Para cada test, compararemos en que puesto del ranking queda nuestro equipo y cuantos partidos gano.\\
FALTA CODIGO Y UNA TABLA CON LOS RESULTADOS COMPARADOS\\