\section{Desarrollo}

\subsection{Discusión Teórica}

Para efectuar los análisis del método de la matriz de Colley (CMM), se asume que ambos métodos de resolución: Eliminación Gaussiana (EG) y Factorización de Cholesky (CL) funcionan siempre. 

Es sabido que no todas las matrices admiten CL o el algoritmo EG y la implementación de ambos métodos en este Trabajo Práctico es fiel a los algoritmos presentados, con lo cual si la asunción previa no se cumple, las corridas que dan base a la experimentación serían indeterminadas (cuelgues de máquina o generación de resultados basura).

La justificación de que la asunción se cumple viene dada por el hecho de que la matriz de Colley desde su armado es simetrica definida positiva, como está explicado en el paper ''Colley's Bias Free College Football Ranking Method: The Colley Matrix Explained''  de W. Colley, sección 7.2

\begin{itemize}

\item SDP $\implies$ A inversible y todas las submatrices principales de A inversibles $\implies$ A tiene factorización LU

\item La factorización LU tal que L tenga 1s en su diagonal es unica

\end{itemize}

Si es única, debe ser la factorización obtenida por el algoritmo de elimincación Gaussiana por motivos constructivos, dado que esta tiene 1s en su diagonal.

\vspace{1em}

Propiedad: Sea \(A \in \mathbb{R}^{n \times n}\) Simétrica Definida Positiva \(\implies A\) No singular $land$ todas las submatrices principales de $A$ No singulares.

Demostración: Por absurdo:

\begin{itemize}

\item Supongamos que $A$ es Singular \(\implies \exists \tilde{x} \) tq \(A\tilde{x} = 0 \) con \(\tilde{x} \neq 0 \)

\begin{equation}
\tilde{x}^{T}\underbrace{A\tilde{x}}_{0} = 0 \text{\hspace{3em} Abs! porque $A$ es S.D.P \hspace{1em}} (x^{T}Ax > 0 \hspace{1em}\forall x \neq 0)
\end{equation}


Luego $A$ es No Singular

\item Sea $A_{k}$ la submatriz principal $\mathbb{R}^{k\times k}$ de $A \in \mathbb{R}^{n \times n}$

Supongo que $A_{k}$ es singular (No inversible) $\implies \exists \bar{x_k}$ tq $A_k \bar{x_k} = 0$ con $\bar{x_k} \neq 0$

\[\text{Sea } \bar{x_{k}} \in \mathbb{R}^{k} \text{tq } A_{k} \bar{x_{k}} = 0  \land \bar{x_{k}} \neq 0 \]

\[ \text{Sea } \bar{x} = \begin{pmatrix} \bar{x_{k}} \\ 0 \\ \vdots \\ 0 \end{pmatrix} \in \mathbb{R}^{n} \]

\begin{equation}
 \bar{x}^{T} A \bar{x} = \begin{pmatrix} \bar{x_{k}} & 0 & \cdots & 0 \end{pmatrix}
  \begin{blockarray}{ccccc|cccc}
    \begin{block}{(ccccc|cccc@{\hspace*{5pt}})}
	\BAmulticolumn{5}{c|}{\multirow{5}{*}{$A_{k}$}}&\BAmulticolumn{4}{c}{\multirow{5}{*}{}}\\
    &&&&&&&\\
    &&&&&&&\\
    &&&&&&&\\
    &&&&&&&\\
    \cline{1-5}% don't use \hline
    \BAmulticolumn{5}{c}{\multirow{4}{*}{}}&\BAmulticolumn{4}{c}{\multirow{4}{*}{}}\\
    \BAmulticolumn{5}{c}{\multirow{4}{*}{}}&\BAmulticolumn{4}{c}{\multirow{4}{*}{}}\\
    \BAmulticolumn{5}{c}{\multirow{4}{*}{}}&\BAmulticolumn{4}{c}{\multirow{4}{*}{}}\\
    \BAmulticolumn{5}{c}{\multirow{4}{*}{}}&\BAmulticolumn{4}{c}{\multirow{4}{*}{}}\\
    \end{block}
  \end{blockarray}
  \begin{pmatrix} \bar{x_{k}} \\ 0 \\ \vdots \\ 0 \end{pmatrix} =
  \begin{pmatrix} \bar{x_{k}} & 0 & \cdots & 0 \end{pmatrix}   \  \underbrace{\begin{pmatrix} 0 \\ \vdots \\ 0 \\ * \\ \vdots \\ * \end{pmatrix}}_{A \bar{x}} = 0
\end{equation}
Los primeros $k$ elementos del vector $A \bar{x}$ son iguales a $0$ dado que haciendo el producto por bloques de $A$ con $\bar{x}$ por cada fila de $A$ las $n-k$ ultimas columnas se anulan al ser multiplicadas por los $n-k$ ultimos ceros de $\bar{x}$ y el primer bloque de dimension $k \times 1$ del vector resultado ( $ = A_k \bar{x_{k}} )$ es cero por hipótesis.

El producto $\bar{x}^{T} (A \bar{x} )$ es igual a cero dado que los primeros $k$ elementos de $\bar{x}^{T}$ son multiplicados por los primeros $k$ ceros del producto $A \bar{x}$ y los $n-k$ ceros de $\bar{x}^{T}$ anulan los $n-k$ ultimos elementos de $A \bar{x}$

\[\text{Abs! pues } A \text{es SDP y } \bar{x} = \begin{pmatrix} \bar{x_{k}} \\ 0 \\ \vdots \\ 0 \end{pmatrix} \neq 0\]

O sea toda $A \in \mathbb{R}^{n \times n}$ SDP tiene todas sus submatrices principales no singulares y $A$ es no singular \qed

\end{itemize}

Propiedad: Sea $A \in \mathbb{R}^{n \times n}$ $A$ inversible (no singular), si las submatrices principales de $A$ son no singulares $\implies$ $A = L U$

Demostración: (x inducción)

\begin{itemize}
	\item $n = 2$
	
	Las hipótesis son:
	\[ A = \begin{pmatrix} a_{1,1} & a_{1,2} \\ a_{2,1} & a_{2,2} \end{pmatrix} \hspace{5em}A \text{ No Singular}\]
	\[ A_{1,1} = \begin{pmatrix} a_{1,1} \end{pmatrix} \hspace{5em}A_{1,1} \text{ No Singular} \]
	
	Ya que la submatriz principal de orden $1$ es $ A_{1,1} = \begin{pmatrix} a_{1,1} \end{pmatrix} $ (la única submatriz principal). Para ser No Singular debe cumplirse $a_{1,1} \neq 0$
	
	Al ser $a_{1,1} \neq 0$ puedo aplicar el algoritmo EG sobre $A$
	\[F_2 - \frac{a_{2,1}}{a_{1,1}} F_1 \]
	Con esto consigo una factorización $LU$: $A = LU$ de la forma usual (algoritmo EG)
	
	\[ L = \begin{pmatrix} 1 & 0 \\ \frac{a_{2,1}}{a_{1,1}} & 1 \end{pmatrix} \hspace{3em}
	U = \begin{pmatrix} a_{1,1} & a_{1,2} \\ 0 & a_{2,2} - \frac{a_{2,1}}{a_{1,1}} a_{1,2} \end{pmatrix}
	\] 
	
	Luego para el caso base $n = 2$ se verifica la tésis.
	
	\item $n \implies n+1$
	
	\[A = 
	\begin{blockarray}{cccc|c}
    \begin{block}{(cccc@{\hspace*{45pt}}|c@{\hspace*{5pt}})}
    \BAmulticolumn{4}{c|}{\multirow{4}{*}{$A^{(n)}$}}&\BAmulticolumn{1}{c}{\multirow{4}{*}{$c_{n+1}$}}\\
	&&&&\\
    &&&&\\
    &&&&\\
    \cline{1-5}% don't use \hline
    \BAmulticolumn{4}{c|}{f_{n+1}}&a_{(n+1)(n+1)} \\
    \end{block}
  \end{blockarray}
  \hspace{5em} A \in \mathbb{R}^{(n+1)\times(n+1)} 
  \]
  \[ A^{(n)} = L^{(n)}U^{(n)}\]
  Propongo
  \[ L =
  \begin{blockarray}{cccc|c}
    \begin{block}{(cccc@{\hspace*{45pt}}|c@{\hspace*{5pt}})}
    \BAmulticolumn{4}{c|}{\multirow{4}{*}{$L^{(n)}$}}&0 \\
	&&&& 0\\
    &&&& \vdots\\
    &&&& 0\\
    \cline{1-5}% don't use \hline
    \BAmulticolumn{4}{c|}{l_{n+1}}& 1 \\
    \end{block}
  \end{blockarray}
  \hspace{7em}
  U =
  \begin{blockarray}{cccc|c}
    \begin{block}{(cccc|c@{\hspace*{5pt}})}
    \BAmulticolumn{4}{c|}{\multirow{4}{*}{$U^{(n)}$}}&\BAmulticolumn{1}{c}{\multirow{4}{*}{$u_{n+1}$}}\\
	&&&&\\
    &&&&\\
    &&&&\\
    \cline{1-5}% don't use \hline
    0 & 0 & \cdots & 0 & u_{(n+1)(n+1)} \\
    \end{block}
  \end{blockarray}
  \]
  
  \[ A = L^{(n+1)} U^{(n+1)} \]
  
  \[
  \begin{blockarray}{cccc|c}
    \begin{block}{(cccc@{\hspace*{45pt}}|c@{\hspace*{5pt}})}
    \BAmulticolumn{4}{c|}{\multirow{4}{*}{$A^{(n)}$}}&\BAmulticolumn{1}{c}{\multirow{4}{*}{$c_{n+1}$}}\\
	&&&&\\
    &&&&\\
    &&&&\\
    \cline{1-5}% don't use \hline
    \BAmulticolumn{4}{c|}{f_{n+1}}&a_{(n+1)(n+1)} \\
    \end{block}
  \end{blockarray}
  =
  \begin{blockarray}{cccc|c}
    \begin{block}{(cccc@{\hspace*{45pt}}|c@{\hspace*{5pt}})}
    \BAmulticolumn{4}{c|}{\multirow{4}{*}{$L^{(n)}$}}&0 \\
	&&&& 0\\
    &&&& \vdots\\
    &&&& 0\\
    \cline{1-5}% don't use \hline
    \BAmulticolumn{4}{c|}{l_{n+1}}& 1 \\
    \end{block}
  \end{blockarray}
  \hspace{1em}
  \begin{blockarray}{cccc|c}
    \begin{block}{(cccc|c@{\hspace*{5pt}})}
    \BAmulticolumn{4}{c|}{\multirow{4}{*}{$U^{(n)}$}}&\BAmulticolumn{1}{c}{\multirow{4}{*}{$u_{n+1}$}}\\
	&&&&\\
    &&&&\\
    &&&&\\
    \cline{1-5}% don't use \hline
    0 & 0 & \cdots & 0 & u_{(n+1)(n+1)} \\
    \end{block}
  \end{blockarray}
  \]
  
  Resolviendo Por bloques tenemos:
  
  \begin{equation} \label{demo:Block1}
  	L^{(n)} U^{(n)} = A^{(n)} 
  \end{equation} 
  \begin{equation} \label{demo:Block2}
	L^{(n)} u_{n+1} = c_{n+1} 
  \end{equation}
  \begin{equation} \label{demo:Block3}
	f_{n+1} = l_{n+1} + U^{(n)}
  \end{equation}
  \begin{equation} \label{demo:Block4}
	l_{(n+1)} u_{(n+1)} + u_{(n+1)(n+1)} = a_{(n+1)(n+1)}
  \end{equation}
  
  Para poder armar constructivamente $L$ y $U$ necesitamos a partir de esto obtener los bloques $u_{n+1} \in \mathbb{R}^{n \times 1}$ , $l_{n+1} \in \mathbb{R}^{1 \times n}$ y $u_{(n+1)(n+1)} \in \mathbb{R}^{1 \times 1}$.
  
  Como $L^{(n)}$ en (\ref{demo:Block2}) es triangular inferior no singular (por hipotesis), $u_{n+1}$ puede resolverse facilmente. Asimismo en (\ref{demo:Block3}) $U^{(n)}$ es triangular superior inversible (por hipotesis) y $f_{n+1}$ tambien queda determinado.
  
	Finalmente en (\ref{demo:Block4}) $u_{(n+1)(n+1)}$ queda determinado resolviendo un producto de matrices y despejando con lo que conseguimos $L$ y $U$ tales que $A = L U$ siempre que tanto $A \in \mathbb{R}^{n \times n}$ como sus submatrices principales sean no inversibles $\forall n \in \mathbb{N}$
\qed

\end{itemize}

Ya demostramos que por ser $A$ SDP, tanto $A$ como todas sus submatrices principales son inversibles, luego $A$ tiene factorizacion $LU$. Nos falta demostrar que esta factorizacion es unica bajo ciertas condiciones.

Queremos llegar a que la matriz de Colley admite el metodo EG, sabemos que el algoritmo nos obtiene dos matrices $L$ y $U$ donde $L$ es una matriz triangular inferior con unos en la diagonal y $U$ es una matriz triangular superior.
Intentemos probar que si $A$ es No Singular (inversible), con factorizacion $A = LU$ y exigimos que $L$ tenga unos en la diagonal, bajo estas condiciones la factorizacion $LU$ es unica.

Hipotesis:
\begin{itemize}

	\item $A$ es inversible
	\item $L$ es una matriz triangular inferior con unos en la diagonal
	\item $L'$ es una matriz triangular inferior con unos en la diagonal
	
	\begin{equation} \label{demo2:1}
	A = LU = L'U'
	\end{equation}

	Premultiplicando a ambos lados por $L^{-1}$
	\begin{equation} \label{demo2:2}
		U = L^{-1} L' U'
	\end{equation}
	Post-multiplicando a ambos lados por $U'^{-1}$
	\begin{equation} \label{demo2:3}
		\underbrace{U U'^{-1}}_{\text{t.s}} = \underbrace{L^{-1} L'}_{\text{t.i.}} = D
	\end{equation}
	La única manera que una matriz triangular inferior sea igual a una matriz diagonal superior es que la matriz sea diagonal (producto de matrices t.i es t.i, producto de matrices t.s es t.s.).
	Luego resolviendo una de las igualdades (pre-multiplicando por la matriz necesaria) tenemos:
	\begin{equation} \label{demo2:4}
		L' = L D
	\end{equation}
	Observando el producto de matrices
	\begin{equation} \label{demo2:5}
	\underbrace{
		\left(
		\begin{array}{ccccc}
		1 \\
		& 1 & &\textbf{\huge0}\\
		& & \ddots \\
		& \text{\huge $l'$} & & \\
		& & & & 1
		\end{array}
		\right)
	}_{\textbf{\huge $L'$}}
		=
	\underbrace{
		\left(
		\begin{array}{ccccc}
		1 \\
		& 1 & &\textbf{\huge0}\\
		& & \ddots \\
		& \text{\huge $l$} & & \\
		& & & & 1
		\end{array}
		\right)
	}_{\textbf{\huge $L$}}
	\underbrace{
		\left(
		\begin{array}{ccccc}
		d_{11} \\
		& d_{22} & & \textbf{\huge0} \\
		& & \ddots \\
		& \textbf{\huge0}\\
		& & & & d_{nn}
		\end{array}
		\right)
		}_{\textbf{\huge $D$}}
	\end{equation}
	El producto de $L$ por $D$ es una matriz diagonal y está igualada a una t.i con unos en la diagonal, luego $D = I$, luego reemplazando $D$ en (\ref{demo2:3})
%	\[ D = I \]
\begin{equation} \label{demo2:6}
	L^{-1} L' = I \hspace{2em} \land \hspace{2em} U U'^{-1} = I
\end{equation}
	Con lo cual pre-multiplicando ambos miembros por $L$ en la primera y post-multiplicando por $U'$ en la segunda:
\begin{equation}
	 L' = L  \hspace{2em} \land \hspace{2em} U = U'
\end{equation}

Luego la factorizacion $LU$ tal que $L$ tenga unos en su diagonal es unica.

\qed
\end{itemize}





\subsection{Heurística para maximizar posición minimizando partidos ganados}

Se propone como heurística: ''Jugar con los equipos más fuertes de la tabla''. La idea es que si juego contra equipos de ranking alto y pierdo, mi ranking no disminuye tanto como si jugara con equipos más débiles, y en el caso de ganarles mi ranking aumentaría más que si le gano a equipos débiles. (Experimentar)
